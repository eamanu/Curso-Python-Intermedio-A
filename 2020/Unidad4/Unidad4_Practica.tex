\documentclass[12pt, spanish]{article}
\usepackage[margin=1in]{geometry}
\usepackage[spanish]{babel}
\selectlanguage{spanish}
\usepackage[utf8]{inputenc}
\setlength{\parindent}{2cm}
\usepackage{hyperref}
\usepackage{graphicx}
\usepackage{float}

% Title Page
\title{Trabajo Práctico Unidad 4}
\author{Python Intermedio - UNLAR 2020}


\begin{document}
\maketitle

\section{Ejercicios}

Resolver el ejercicio en la mayor cantidad posible. Enviarlos vía Gitlab, en la carpeta
\textit{2020/Unidad4/TrabajoPractico/Nombre\_Apellido/}

Desarrollar una herramienta que me permita tener el árbol de dependencia de una módulo o 
herramienta de Python. Esto es muy importante cuando se hace mantenimiento de un paquete
en una distribución, o cuando se mantiene un modulo de Python.

Para desarrollar esta herramienta van a necesitar recibir como argumento el nombre del paquete
(ver último ejercicio de la unidad 2). Además van a necesitar, descargar el wheel, usando pip,
para ellos pueden ejecutar, usando subprocess.Popen, el comando:

$$python3 -m pip wheel poetry -vvv --no-deps --no-cache-dir --disable-pip-version-check$$

Y podran sacar información del \textit{wheel} usando los modulos: \textit{pkginfo, pkg\_resource,
distlib}

Con los datos que obtengan con esos módulos, podrán seguir repitiendo el proceso.

La idea es que utilicen procesos para acelerar:

\begin{itemize}
	\item La descarga de los wheels.
	\item La búsqueda de los requerimientos.
	\item El armado de la estructura de datos final para presentar el árbol de dependencia.
\end{itemize}

Hacer esta herramienta, no es una tarea sencilla, por lo que se pide realizar lo mayor
posible, y también es válido dejar partes del código definida (sin implementar).

Podrán encontrar inspiración aquí \url{https://github.com/wimglenn/johnnydep}

\end{document}          
